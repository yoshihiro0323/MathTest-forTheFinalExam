\documentclass[a3paper,twocolumn,fleqn]{ltjsarticle}
\usepackage[top=17truemm,bottom=17truemm,left=11truemm,right=11truemm]{geometry}
\usepackage{tikz}
\usetikzlibrary{intersections,calc,arrows.meta}
\usepackage{amssymb}
\usepackage{enumerate}
\usepackage{amsmath}
\usepackage{fancyhdr}
\usepackage{luatexja-otf}
\usepackage[no-math]{luatexja-fontspec}
\setmainjfont[BoldFont=HiraMinProN-W6]{HiraMinProN-W3}
\setsansjfont[BoldFont=HiraginoSans-W6]{HiraginoSans-W3}
\setsansfont{Helvetica}
\setlength{\columnseprule}{0.4pt}
\setlength{\columnsep}{2\zw}
\pagestyle{empty}
\begin{document}
\section{問題}
関数$y=e^x$で表される曲線$C_1$上を動く点P$(t,e^t)$に対し,点Pで曲線$C_1$の上側で接する半径$1$の円$C$を考え,その中心を点O$^\prime$とする.線分PQが円$C$の直径になるような点Qとする.以下の問いに答えよ.
\begin{enumerate}[(1)]
    \item 点O$^\prime$の座標を$t$を用いて表せ.
    \item 点Pを$\displaystyle 0\leqq t\leqq \frac{\log3}{2} $の範囲で動かしたとき,点O$^\prime$が通る軌跡の曲線の長さを求めよ.
    \item $t$が実数全体を動くときの点Qが描く曲線を曲線$C_2$とする.また,$t=0$での点Qの$x$座標の値を$a$,$\displaystyle t=\frac{\log3}{2}$での点Qの$x$座標の値を$b$とするとき,曲線$C_1$と$x$軸,$x=a$,$x=b$で囲まれた部分の面積を求めよ.
    \item 点Pを$\displaystyle 0\leqq t\leqq \frac{\log3}{2} $の範囲で動かしたとき,円$C$が通過する領域の面積を求めよ.
\end{enumerate}

\section{解答}
\begin{enumerate}[(1)]
    \item $y=e^x$のとき,$y'=e^x$より,点Pにおける法線の傾きは$\displaystyle -\frac{1}{e^t}$ .これは直線PO$^\prime$の傾きでもあるので,$\left| \overrightarrow{\mbox{PO}^\prime} \right| = 2$ と,円$C$は$C_1$の上側で接することより,\\
    $\displaystyle \overrightarrow{\mbox{PO}^\prime} = \left( -\frac{e^t}{\sqrt{e^{2t}+1}} , \frac{1}{\sqrt{e^{2t}+1}}\right)$.
    P$(t,e^t)$より,$\displaystyle \mbox{O}^\prime \left( t - \frac{e^t}{\sqrt{e^{2t}+1}} , e^t + \frac{1}{\sqrt{e^{2t}+1}}\right)$\\
    \item 
    \begin{align*}
        \frac{dx}{dt} &= 1-\frac{e^t}{{\left( e^{2t}+1 \right)}^\frac{3}{2}} & \frac{dy}{dt} &= e^t - \frac{e^{2t}}{{\left( e^{2t}+1 \right)}^\frac{3}{2}}
    \end{align*}
    求める曲線の長さを$\ell$として,
    \begin{align*}
        \ell &= \int_{0}^{\frac{\log3}{2}}\sqrt{\left( \frac{dx}{dt} \right)^2 + \left( \frac{dy}{dt} \right)^2} dt\\
        &= \int_{0}^{\frac{\log3}{2}} \left(1 - \frac{e^t}{\left(e^{2t} + 1\right)^\frac{3}{2}}\right)\sqrt{1+e^{2t}} dt\\
        &= \int_{0}^{\frac{\log3}{2}} \left(\sqrt{1+e^{2t}} - \frac{e^t}{e^{2t} + 1}\right) dt
    \end{align*}
    $e^t = \tan\theta$と置換すると,$\displaystyle 0\leqq t\leqq \frac{\log3}{2} $より,$\displaystyle \frac{\pi}{4} \leqq\theta\leqq\frac{\pi}{3}$
    \begin{align*}
        e^t &= \frac{d\theta}{dt}\frac{1}{\cos^2\theta}\\
        \Leftrightarrow dt &= \frac{d\theta}{e^t \cos^2\theta} = \frac{d\theta}{\tan\theta\cos^2\theta}= \frac{d\theta}{\sin\theta\cos\theta}\\
    \end{align*}
    よって,
    \begin{align*}
        \int_{0}^{\frac{\log3}{2}} \sqrt{1+e^{2t}} &= \int^{\frac{\pi}{3}}_{\frac{\pi}{4}} \frac{1}{\sin\theta \cos^2\theta} d\theta\\
        &= \int^{\frac{\pi}{3}}_{\frac{\pi}{4}} \left(\frac{\sin\theta}{\cos^2\theta} + \frac{1}{\sin\theta}\right) d\theta\\
        &= \int^{\frac{\pi}{3}}_{\frac{\pi}{4}} -\frac{\left(\cos\theta\right)^\prime}{\cos^2\theta} d\theta + \int^{\frac{\pi}{3}}_{\frac{\pi}{4}} \frac{d\theta}{\sin\theta}\\
        &= \left[\frac{1}{\cos\theta}\right]^{\frac{\pi}{3}}_{\frac{\pi}{4}} + \int^{\frac{\pi}{3}}_{\frac{\pi}{4}} \frac{d\theta}{\sin\theta} = 2 -\sqrt{2} + \int^{\frac{\pi}{3}}_{\frac{\pi}{4}} \frac{d\theta}{\sin\theta}\\
        \int^{\frac{\pi}{3}}_{\frac{\pi}{4}} \frac{1}{\sin\theta} d\theta  &= \int^{\frac{\pi}{3}}_{\frac{\pi}{4}} \frac{\sin\theta}{1-\cos^2\theta} d\theta\\
        &= \frac{1}{2}\int^{\frac{\pi}{3}}_{\frac{\pi}{4}} \left(\frac{\sin\theta}{1-\cos\theta}+\frac{\sin\theta}{1+\cos\theta}\right) d\theta\\
        &= \frac{1}{2}\left[\log\left(\frac{1-\cos\theta}{1+\cos\theta}\right)\right]^{\frac{\pi}{3}}_{\frac{\pi}{4}} = -\frac{1}{2}\log\left(9-6\sqrt{2}\right)\\
        \int_{0}^{\frac{\log3}{2}} \frac{e^t}{e^{2t}+1} dt &= \int^{\frac{\pi}{3}}_{\frac{\pi}{4}} d\theta = \Big[\theta\Big]^{\frac{\pi}{3}}_{\frac{\pi}{4}} = \frac{\pi}{12}\\
    \end{align*}
    以上より,
    \begin{align*}
        \ell &= 2 -\sqrt{2} -\frac{1}{2}\log\left(9-6\sqrt{2}\right) - \frac{\pi}{12}
    \end{align*}
    \item (1)の結果より,\\
    Q$\displaystyle \left( t - \frac{2e^t}{\sqrt{e^{2t}+1}} , e^t + \frac{2}{\sqrt{e^{2t}+1}}\right)$.$\displaystyle f(t) = t - \frac{2e^t}{\sqrt{e^{2t}+1}}$とおくと, \\
    \begin{align*}
        a & = f(0) = -\sqrt{2} & b & = f\left(\frac{\log3}{2}\right) = \frac{\log3}{2}-\sqrt{3}
    \end{align*}
    また,
    \begin{align*}
        \frac{dx}{dt} &= 1-2\frac{e^t \sqrt{e^{2t}+1} - e^t \frac{e^{2t}}{\sqrt{e^{2t}+1}}}{e^{2t}+1} = 1-\frac{2e^t}{{\left( e^{2t}+1 \right)}^\frac{3}{2}}
    \end{align*}
    よって,
    \begin{align*}
        \int_{b}^{a} y dx &= \int_{0}^{\frac{\log3}{2}} \left( e^t + \frac{2}{\sqrt{e^{2t}+1}}\right)\left( 1-\frac{2e^t}{{\left( e^{2t}+1 \right)}^\frac{3}{2}} \right) dt \\
        &= \int_{0}^{\frac{\log3}{2}} \left(e^t + \frac{2}{\sqrt{e^{2t}+1}} - \frac{2e^{2t}}{{\left( e^{2t}+1 \right)}^\frac{3}{2}} - \frac{4e^t}{\left( e^{2t}+1 \right)^2}\right)dt\\
    \end{align*}
    ここで,
    \begin{align*}
        \int_{0}^{\frac{\log3}{2}} e^t &= \Big[e^t \Big]_{0}^{\frac{\log3}{2}} = \sqrt{3} - 1 \\
        \int_{0}^{\frac{\log3}{2}} \frac{2e^{2t}}{{\left( e^{2t}+1 \right)}^\frac{3}{2}} dt &= \int_{0}^{\frac{\log3}{2}} \frac{\left( e^{2t}+1 \right)^\prime}{{\left( e^{2t}+1 \right)}^\frac{3}{2}} dt\\
        &= \left[-\frac{2}{\sqrt{e^{2t}+1}}\right]_{0}^{\frac{\log3}{2}} = \sqrt{2} - 1
    \end{align*}
    (2)と同様に置換すると,
    \begin{align*}
        \int_{0}^{\frac{\log3}{2}} \frac{2}{\sqrt{e^{2t}+1}} dt &= \int^{\frac{\pi}{3}}_{\frac{\pi}{4}} \frac{2}{\sin\theta} d\theta = \log\left(9-6\sqrt{2}\right)\\
        \int_{0}^{\frac{\log3}{2}} \frac{4e^t}{\left( e^{2t}+1 \right)^2} dt &= 4\int^{\frac{\pi}{3}}_{\frac{\pi}{4}} \cos^2\theta d\theta\\
        &= 2\int^{\frac{\pi}{3}}_{\frac{\pi}{4}} \left(1+\cos2\theta\right) d\theta\\
        &= \Big[2\theta + \sin2\theta\Big]^{\frac{\pi}{3}}_{\frac{\pi}{4}}\\
        &= \frac{\pi}{6} + \frac{\sqrt{3}}{2} - 1
    \end{align*}
    以上より,求める面積を$S$として,
    \begin{align*}
        S &= \left(\sqrt{3} - 1\right) +\log\left(9-6\sqrt{2}\right) -\left(\sqrt{2} - 1\right) -\left( \frac{\pi}{6} + \frac{\sqrt{3}}{2} - 1\right)\\
        &= 1 -\sqrt{2} + \frac{\sqrt{3}}{2} +\log\left(9-6\sqrt{2}\right)- \frac{\pi}{6}
    \end{align*}
    \item 求める領域は以下の通り.\\
    \begin{tikzpicture}
        \draw[->,>=stealth,semithick](-4,0)--(2,0)node[above]{$x$};%x軸
        \draw[->,>=stealth,semithick](0,-0.5)--(0,4.5)node[right]{$y$};%y軸
        \draw(-1.1827,0)node[below]{$\displaystyle\frac{\log3}{2} - \sqrt{3}$};
        \draw(0,2.7320)node[right]{$1+\sqrt{3}$};
        \draw[densely dashed](-1.1827,0)--(-1.1827,2.7320)--(0,2.7320);
        \draw[densely dashed](0.5493,1.7320)--(-1.1827,2.7320);
        \draw(-1.4142,0)node[above left]{$-\sqrt{2}$};
        \draw[densely dashed](-1.4142,0)--(-1.4142,2.4142)--(0,2.4142);
        \draw(0,2.4142)node[right]{$1+\sqrt{2}$};
        \draw[densely dashed](0,1)--(-1.4142,2.4142);
        \draw(0,0)node[below left]{O};%原点
        \draw(0,1)node[below right]{$1$};
        \draw(0.5493,0)node[below]{$\displaystyle \frac{\log3}{2}$};
        \draw[densely dashed](0.5493,0)--(0.5493,1.7320)--(0,1.7320);
        \draw(0,1.73202)node[left]{$\sqrt{3}$};
        \draw[domain=-4:1.5,name path=C1]plot(\x,{exp(\x)})node[right]{$C_1:y=e^x$};
        \draw(-0.7071,1.7071)circle(1);
        \draw(-0.3167,2.2320)circle(1);
        \draw(-1.4142,2.4142)to[out=45,in=60](-1.1827,2.7320);
        %\draw[domain=0:2,name path=C2,variable=\t]plot({\t - {{2*exp(\t)}/{sqrt({exp(2*\t)+1})}}},{\t})node[above right]{$C_1:y=e^x$};
    \end{tikzpicture}\\
    点Pにおける法線の傾きは$\displaystyle -\frac{1}{e^t}$ なので,\\
    $t=0$のとき,傾きは$-1$で,$\displaystyle t=\frac{\log3}{2}$のとき,傾きは$\displaystyle -\frac{1}{\sqrt{3}}$ である.\\
    以上より,求める面積を$S^\prime$として,
    \begin{align*}
        S^\prime &= 1-\sqrt{2} + \frac{\sqrt{3}}{2} +\log\left(9-6\sqrt{2}\right)- \frac{\pi}{6} - \frac{1}{2}\left(1+1+\sqrt{2}\right)\sqrt{2} + \frac{1}{2}\left(\sqrt{3}+1+\sqrt{3}\right)\sqrt{3}\\
        &- \int_{0}^{\frac{\log3}{2}} e^x dx + \pi\\
        &= 4 -2\sqrt{2} -\log\left(9-6\sqrt{2}\right) + \frac{5}{6}\pi
    \end{align*}
    \fbox{別解}\\
    (2)の結果より,
    \begin{align*}
        S^\prime &= 2\left(2 -\sqrt{2} -\frac{1}{2}\log\left(9-6\sqrt{2}\right) - \frac{\pi}{12}\right) + \pi\\
        &= 4 -2\sqrt{2} -\log\left(9-6\sqrt{2}\right) + \frac{5}{6}\pi
    \end{align*}
\end{enumerate}
\section{感想・考察}
円の内部や外部を円が接しながら移動する問題を応用した.曲線$C_1$を表す関数を$y=x^2$などの関数での問題にしようとしたが,高校数学範囲で積分できない式が出てきてしまったので,$y=e^x$とし,曲線の長さを求める問題を追加した.\\
また,点Qの座標を$t$で表したときに複雑にならないよう,曲線$C_1$の下側ではなく上側を円が動くようにした.加えて,$e^t = \tan\theta$の置換ができるように範囲を設定した.\\
計算結果が綺麗になるように改善すればより良い問題になるだろう.
\end{document}